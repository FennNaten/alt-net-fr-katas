\documentclass[10pt,a4paper,final]{article}
\usepackage[utf8]{inputenc}
\usepackage{amsmath}
\usepackage{amsfonts}
\usepackage{amssymb}
\usepackage{algorithm}
\usepackage{algorithmic}
\begin{document}
Soit $n$ un nombre pair de digits.
Soit l'ensemble $$E_n=\{10^{\frac{n}{2}-1},...,10^{\frac{n}{2}}-1\}^2$$
Par exemple $E_4 = \{10,11,12,...,99\}^2$.

On considère le sous ensemble $$C_n = \{(i,j) \in E_n, i\geq j\}$$ 

On munit $C_n$ de la relation d'ordre lexicographique, ie :
$(i,j) < (k,l)$ ssi $i<k$ ou ($i=k$ et $j<l$).

Soit $(U_k)$ la suite décroissante des éléments de $C_n$.

On peut construire $(U_k)$ avec l'algorithme suivant : 
\begin{algorithm}
\caption{Parcourir $C_n$ en décroissance}
\begin{algorithmic} 
\STATE $k \leftarrow 0$
\FOR{$i=10^{\frac{n}{2}}-1$;$i>=10^{\frac{n}{2}-1}$;$i=i-1$}
\FOR{$j=i$;$j>=10^{\frac{n}{2}-1}$;$j=j-1$}
\STATE $U_k = (i,j)$
\STATE $k=k+1$
\ENDFOR
\ENDFOR
\end{algorithmic}
\end{algorithm}

On note $f : C_n \longrightarrow \mathbb{N}$ l'application qui pour un couple $(i,j)$ donne l'indice $k$ correspondant tel que $U_k = (i,j)$.

On note $V_k$ le produit des éléments du couble $U_k$, ie :
$U_k = (i,j) \Rightarrow V_k = ij$.

On note $D_k$ le premier digit de $V_k$, ex :
$V_k=4321 \Rightarrow D_k=4$.

On remarque que pour tout couple de type $(i,i)$, on a :

Pour tout $k$ tel que $k > f(i,i)$, alors $i^2>V_k$.

Cela implique :

Pour tout $k$ tel que $k > f(i,i)$, alors $D_{f(i,i)}\geq D_k$


Cette constatation nous permet d'améliorer l'algorithme brute force pour trouver les nombres vampires :

Il suffit de monitorer les digits $D_f(i,i)$ et d'exclure les couples $(k,l)<(i,i)$ si $k$ et $l$ sont des nombres dont tout les digits sont strictement supérieurs à $D_{f(i,i)}$, ce test peut être fait de manière efficace ($O(1)$) en monitorant la position dans le couple du digit de poids le plus fort qui permet de ne pas exclure ce couple.
Par ailleurs, si $D_{f(i,i)} = 0$, alors le nombre de digit représentant $V_f(i,i)$ est inférieur à $n$ et l'algorithme s'arrête.
\newline

Ci dessous les nombres d'itérations pour le brute force comparé au nombre d'itérations en excluant les couples selon notre critère :
\newline

\begin{tabular}{lcccc}

  \hline
    & $n = 4$ & $n=6$ & $n=8$ & $n=10$ \\
  notre algo & $3426$ & $365140$ & $37285158$ & $3766116479$ \\
  brute force & $4005$ & $404550$ & $40495500$ & $4049955000$ \\
  rapport & $85.5\%$ & $90.2\%$ & $92.0\%$ & $92.9\%$\\
  \hline
  \end{tabular}



  





\end{document}

